% !TeX spellcheck = fr_FR
\documentclass[a4paper]{report}


%% Language and font encodings
\usepackage[french]{babel}
\usepackage[utf8]{inputenc}
\usepackage[T1]{fontenc}

%% Sets page size and margins
\usepackage[a4paper,top=3cm,bottom=2cm,left=3cm,right=3cm,marginparwidth=1.75cm]{geometry}

%% Useful packages
\usepackage{graphicx}
\usepackage[colorinlistoftodos]{todonotes}
\usepackage{amsmath,amsthm,amssymb}
\usepackage[colorlinks=true, allcolors=blue]{hyperref}
\usepackage{listings}
\usepackage{color}
\usepackage{titlesec}
\definecolor{dkgreen}{rgb}{0,0.6,0}
\usepackage{textcomp}
\lstset{frame=tb,
  language=Python,
  aboveskip=3mm,
  belowskip=3mm,
  showstringspaces=false,
  columns=flexible,
  basicstyle={\ttfamily},
  numbers=none,
  numberstyle=\color{gray},
  keywordstyle=\color{blue},
  commentstyle=\color{dkgreen},
  stringstyle=\color{dkgreen},
  breaklines=true,
  breakatwhitespace=true,
  tabsize=2
}
\setlength{\parskip}{1em}

\titleformat{\chapter}{\normalfont\huge}{\bf\thechapter}{12pt}{\huge\bf}

\title{TB - Gleam}

\author{Antoine Friant}
\begin{document}
\pagenumbering{roman}
\begin{titlepage}

\newcommand{\HRule}{\rule{\linewidth}{0.5mm}} % Defines a new command for the horizontal lines, change thickness here
%----------------------------------------------------------------------------------------
%	LOGO SECTION
%----------------------------------------------------------------------------------------

\includegraphics[width=2in]{logoheig.png}\\ % Include a department/university logo - this will require the graphicx package

\vspace{1in}

\center % Center everything on the page
 
%----------------------------------------------------------------------------------------
%	HEADING SECTIONS
%----------------------------------------------------------------------------------------

\textsc{\LARGE HEIG-VD}\\[1.5cm] % Name of your university/college
\textsc{\Large Gleam - Rapport intermédiaire}\\[0.5cm] % Major heading such as course name

%----------------------------------------------------------------------------------------
%	TITLE SECTION
%----------------------------------------------------------------------------------------

\HRule \\[0.4cm]
{ \huge \bfseries La Terre de nuit vue de l'espace }\\ % Title of your document
\Large\HRule \\[1cm]
\normalsize
%----------------------------------------------------------------------------------------
%	AUTHOR SECTION
%----------------------------------------------------------------------------------------

\begin{minipage}[t]{0.48\textwidth}
\begin{flushleft} \large
Antoine \textsc{Friant} \\
\normalsize
Haute École d'Ingénierie et de Gestion du Canton de Vaud\\
Yverdon-les-Bains, VD, CH
\texttt{antoine.friant@gmail.com}
\end{flushleft}
\end{minipage}
~
\begin{minipage}[t]{0.49\textwidth}
\begin{flushright} \large
% empty page for alignement
\end{flushright}
\end{minipage}\\[1cm]



{\large \today}\\
 

\vfill % Fill the rest of the page with whitespace

\end{titlepage}

\chapter{Cahier des charges}
\section{Résumé du problème}
présenter le sujet
\section{Cahier des charges}
\subsection{Objectifs}
Le TB consiste à explorer les données suivantes :

- Images satellites de nuit
- Population humaine
- Population animale
- Densité végétale
- PIB

Ou toute autre donnée jugée pertinente, dans le but d'entraîner un réseau de neurones capable d'estimer une donnée utile à partir d'une image satellite de la terre de nuit.

La réalisation de ce réseau de neurones est l'objectif de la seconde partie du TB.

L'idéal étant de pouvoir estimer, grâce au machine learning, des informations dont on ne possède pas de données à jour. Et cela à partir d'images satellites de nuit, ou une combinaisons de ces images avec une autre donnée à jour.

\subsection{Moyens}
- l'aide du répondant
- le serveurs de GPU de l'école

\subsection{Outils}
- Images satellites de la terre de nuit au format .png ou .tif.
- Datasets libres
- N'importe quel langages de programmation et librairies jugés adéquats.

\subsection{Livrables}
15 juin 2018 : rapport intermédiaire
27 juillet 2018 : rapport final + réseau de neurones fonctionnel utilisant les images satellites de nuit pour produire une estimation d'une donnée utile



\tableofcontents


\chapter{Résumé}
\pagenumbering{arabic}


\chapter{Introduction}
Les produits d'imagerie satellite sont devenus abondants et largement accessible au cours des vingt dernières années. De nombreux satellites prennent des photographies de la Terre à chaque heure du jour \textit{et de la nuit}. Ces observations nocturnes révèlent des caractéristiques peu évidentes de jour, parfois même cachées. Les routes apparaissent, les villes montrent leurs lumières, même les bateaux de pêche aveuglent les océans avec des projecteurs pour attirer les poissons.

La disponibilité, la résolution et l'uniformité de la qualité de ces données contraste fortement avec le manque de fiabilité d'autres informations géographiques utiles lors de prises de décisions importantes. Par exemple, la densité de la population est une estimation précise en Suisse mais très approximative au Kenya. D'autres mesures intéressantes incluent : la consommation en électricité, les émissions de C0$_2$, la couverture végétale et la présence de faune. Les lumières nocturnes observées depuis l'espace donnent des indications sur chacune de ces mesures alors qu'elles peuvent manquer dans une région à une date donnée.

Le but de ce projet est d'extraire autant d'information que possible de l'imagerie satellite nocturne en utilisant l'apprentissage automatique (\textit{machine learning}) sous la forme de réseau de neurones.

\chapter{Body}


\chapter{Conclusion}


\chapter{Bibliographie}


\chapter{Authentification}


\chapter{Symboles et abréviations}


\chapter{Figures}


\chapter{Annexes}


\end{document}